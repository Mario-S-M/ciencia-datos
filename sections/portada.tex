% portada.tex
% Configuración del fondo solo para la portada
\backgroundsetup{
	scale=1,
	angle=0,
	opacity=0.2,
	contents={\includegraphics[width=\paperwidth,height=\paperheight]{imagenes/fondo_portada.png}}
}
\BgThispage

\begin{center}
	\vspace*{2cm}
	
	\Large\textbf{Tarea 2: Primera Extracción de Datos}
	
	\vspace{1.5cm}
	
	Autor(es):
	
	\vspace{0.5cm}
	
	Luis Fernando Chávez Martínez
	
	Mario Eduardo Sánchez Mejía
	
	
	\vspace{1.5cm}
	
	Docente:
	
	\vspace{0.5cm}
	
	
	Aurelio Amaury Coria Ramirez
	
	\vspace{0.5cm}
	
	\begin{abstract}
		\noindent
		\justifying
		La presente investigación aborda el desarrollo de una alternativa de código abierto a las plataformas comerciales de Business Intelligence, centrándose específicamente en replicar las funcionalidades principales de Power BI utilizando Python. Se analiza la arquitectura necesaria para implementar un sistema completo de análisis y visualización de datos, identificando los componentes críticos para el procesamiento, análisis y presentación de información. La investigación examina las diferentes tecnologías y frameworks disponibles para cada componente del sistema, desde la extracción y transformación de datos hasta la visualización interactiva y el despliegue en producción. Este estudio contribuye al campo de la analítica de datos proporcionando un marco de trabajo para la implementación de soluciones personalizadas de Business Intelligence, beneficiando a organizaciones que requieren mayor flexibilidad y control sobre sus herramientas de análisis de datos. \\
		
		\textbf{Palabras clave:} ETL, Visualización de datos, Python, Desarrollo web, Bases de datos, Machine Learning, DevOps, Dashboards interactivos, Análisis de datos, Interfaz de usuario, Business Intelligence, Código abierto.
	\end{abstract}
\end{center}

% Desactivar el fondo para las siguientes páginas
\clearpage
\backgroundsetup{contents={}}