% portada.tex
% Configuración del fondo solo para la portada
\backgroundsetup{
	scale=1,
	angle=0,
	opacity=0.2,
	contents={\includegraphics[width=\paperwidth,height=\paperheight]{imagenes/fondo_portada.png}}
}
\BgThispage

\begin{center}
	\vspace*{2cm}
	
	\Large\textbf{Sensor Infrarojo (deteccion movimiento)}
	
	\vspace{1.5cm}
	
	Autor(es):
	
	\vspace{0.5cm}
	
	Luis Fernando Chávez Martínez
	
	Mario Eduardo Sánchez Mejía
	
	
	\vspace{1.5cm}
	
	Docente:
	
	\vspace{0.5cm}
	
	Juan Jesús Ruiz Lagunas
	
	\vspace{0.5cm}
	
	\begin{abstract}
		\noindent
		\justifying
		Este trabajo presenta el desarrollo e implementación de un detector de movimiento utilizando un sensor infrarrojo PIR (HC-SR501). El sistema es capaz de detectar la presencia de personas o animales mediante la captación de radiación infrarroja emitida por sus cuerpos. Se describe la configuración del circuito electrónico, la integración con Raspberry Pi y la programación necesaria para detectar movimiento y activar un LED indicador. El proyecto demuestra la aplicación práctica de sensores infrarrojos en sistemas de detección automática, con potenciales aplicaciones en seguridad doméstica, iluminación inteligente y sistemas de ahorro energético.
		
		\vspace{0.5cm}
		
		\textbf{Palabras clave:} Sensor PIR, detector de movimiento, infrarrojo, Raspberry Pi, automatización, HC-SR501, domótica, GPIO, sistema embebido.
		
	\end{abstract}
\end{center}

% Desactivar el fondo para las siguientes páginas
\clearpage
\backgroundsetup{contents={}}