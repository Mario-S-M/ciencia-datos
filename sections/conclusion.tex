\section*{Conclusión}

La implementación del detector de movimiento con sensor infrarrojo PIR HC-SR501 y Raspberry Pi ha demostrado ser un proyecto exitoso que ilustra los principios fundamentales de los sistemas de detección automática. A través de este trabajo, se ha logrado comprender el funcionamiento interno del sensor PIR, sus modos de operación y las consideraciones necesarias para su implementación efectiva.

El sistema desarrollado presenta diversas ventajas como su bajo consumo energético, fácil configuración y alta sensibilidad para detectar presencia humana. Sin embargo, también se identificaron algunas limitaciones inherentes al sensor, como su dificultad para detectar movimiento perpendicular a la superficie del domo y la necesidad de un período de inicialización de aproximadamente un minuto.

Las aplicaciones potenciales de este tipo de sistema son numerosas y abarcan desde la automatización doméstica hasta sistemas de seguridad. La posibilidad de ajustar parámetros como la sensibilidad y el tiempo de retardo mediante los potenciómetros R1 y R2 brinda flexibilidad para adaptar el dispositivo a diferentes entornos y requisitos.

Para trabajos futuros, sería interesante expandir este proyecto incorporando múltiples sensores para cubrir áreas más amplias, integrar comunicación inalámbrica para notificaciones remotas, o combinar el sensor PIR con otras tecnologías como ultrasonido o visión por computadora para aumentar la precisión y reducir los falsos positivos.

En conclusión, este proyecto no solo ha permitido desarrollar habilidades prácticas en electrónica y programación, sino que también ha proporcionado una base sólida para comprender los principios fundamentales de los sistemas de detección automática, elementos esenciales en la creciente industria de la Internet de las Cosas (IoT) y la automatización inteligente.