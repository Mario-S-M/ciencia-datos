\section{Introducción}
\noindent
\justifying
En la era actual de la transformación digital, la capacidad de analizar y visualizar datos de manera efectiva se ha convertido en una necesidad fundamental para las organizaciones. Las herramientas de Business Intelligence (BI) comerciales, como Power BI, han dominado el mercado, ofreciendo soluciones robustas pero a menudo limitadas por su naturaleza propietaria y costos asociados. Esta realidad plantea un desafío significativo para organizaciones que buscan mayor flexibilidad, personalización y control sobre sus procesos de análisis de datos.
El ecosistema de Python, con su rica biblioteca de herramientas de código abierto, presenta una oportunidad única para desarrollar alternativas viables a estas soluciones comerciales. Sin embargo, la implementación de un sistema completo de BI requiere una comprensión profunda de múltiples componentes y tecnologías, desde la gestión de datos hasta la visualización interactiva.
Esta investigación se centra en explorar y documentar la arquitectura necesaria para construir una plataforma de BI utilizando Python, abordando los desafíos técnicos y consideraciones de diseño que surgen en el proceso. El estudio no solo examina las tecnologías individuales disponibles, sino que también analiza cómo estas pueden integrarse efectivamente para crear un sistema cohesivo y funcional.
La relevancia de esta investigación se fundamenta en tres aspectos principales: primero, la creciente necesidad de soluciones de BI personalizables y escalables; segundo, la tendencia hacia la adopción de tecnologías de código abierto en entornos empresariales; y tercero, la importancia de democratizar el acceso a herramientas avanzadas de análisis de datos.
Este trabajo busca contribuir al campo del análisis de datos proporcionando un marco de referencia para desarrolladores y organizaciones que desean implementar sus propias soluciones de BI. La investigación no solo aborda aspectos técnicos, sino que también considera factores prácticos como la escalabilidad, mantenibilidad y seguridad, elementos cruciales para cualquier implementación en un entorno de producción.