\section{Introducción}
\noindent
\justifying
Los sensores de movimiento basados en tecnología infrarroja pasiva (PIR) representan una herramienta fundamental en los sistemas de automatización y seguridad modernos. Estos dispositivos aprovechan la capacidad de detectar la radiación infrarroja emitida naturalmente por los cuerpos humanos y animales para identificar su presencia en un área determinada.

El presente trabajo se centra en la implementación de un detector de movimiento utilizando el sensor HC-SR501, un componente ampliamente utilizado en proyectos de electrónica debido a su bajo costo, facilidad de uso y confiabilidad. Este sensor opera detectando cambios en los patrones de radiación infrarroja dentro de su rango de visión, lo que permite activar diversos dispositivos o sistemas cuando se detecta movimiento.

La integración de este sensor con una Raspberry Pi permite crear un sistema inteligente capaz de reaccionar a la presencia humana. En este proyecto específico, el sistema activará un LED cuando detecte movimiento y lo desactivará cuando no haya presencia en el área de detección. Esta funcionalidad básica ilustra el principio operativo que sustenta aplicaciones más complejas como sistemas de iluminación automática, alarmas de seguridad, o contadores de personas.

El objetivo principal de esta práctica es comprender los principios de funcionamiento del sensor PIR, aprender a configurar sus parámetros de sensibilidad y tiempo de activación, e implementar un circuito funcional que demuestre su aplicación práctica. Adicionalmente, se busca familiarizar al estudiante con la programación de GPIO en Raspberry Pi para la lectura de sensores y control de actuadores.