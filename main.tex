\documentclass[12pt]{article}

% Paquetes básicos primero
\usepackage[utf8]{inputenc}
\usepackage[T1]{fontenc}
\usepackage{ragged2e}
\usepackage[spanish]{babel}

% Geometría y diseño
\usepackage{geometry}
\geometry{
	letterpaper,
	left=2.5cm,
	right=2.5cm,
	top=3cm,
	bottom=3cm,
	headheight=60pt,
	headsep=1cm
}

% Paquetes de formato
\usepackage{setspace}
\usepackage{graphicx}
\usepackage{fancyhdr}
\usepackage{xcolor}
\usepackage{tcolorbox}
\usepackage{enumitem}
\usepackage{lastpage}
\usepackage{background}

\usepackage{listings}
\usepackage{inconsolata}
\usepackage{fontawesome}
\usepackage{mdframed}

% Definición de colores para código
\definecolor{lightgray}{gray}{0.95}
\definecolor{darkgray}{gray}{0.3}
\definecolor{mediumgray}{gray}{0.6}

% Estilo para cajas de comando
\newtcolorbox{comando}[1]{
	colback=lightgray,
	colframe=black,
	boxrule=0.5pt,
	arc=2pt,
	title={\faTerminal\ #1},
	fontupper=\ttfamily,
	before skip=10pt,
	after skip=10pt,
	left=5pt,
	right=5pt,
	top=5pt,
	bottom=5pt
}

% Configuración de listings
\lstset{
	basicstyle=\ttfamily\small,
	backgroundcolor=\color{lightgray},
	frame=single,
	framesep=3pt,
	breaklines=true,
	breakatwhitespace=true,
	showstringspaces=false,
	numbers=left,
	numberstyle=\tiny\color{darkgray},
	numbersep=8pt,
	keywordstyle=\color{black}\bfseries,
	commentstyle=\color{mediumgray},
	stringstyle=\color{black},
	captionpos=b,
	framerule=0.5pt,
	rulecolor=\color{black}
}

% Definición de entorno para notas
\newmdenv[
linewidth=0.5pt,
linecolor=black,
topline=false,
rightline=false,
bottomline=false,
leftmargin=10pt,
innerleftmargin=5pt,
skipabove=10pt,
skipbelow=10pt
]{nota}

% Paquetes para referencias
\usepackage[round]{natbib}  % Configuración específica para citas
\bibliographystyle{plainnat}  % Estilo de bibliografía

% URLs y enlaces (hyperref debe ir al final)
\usepackage{url}
\usepackage[hidelinks]{hyperref}

% Configuración del encabezado y pie de página
\pagestyle{fancy}
\fancyhf{}
\renewcommand{\headrulewidth}{0pt}
\renewcommand{\footrulewidth}{0pt}

\definecolor{tecnmBlue}{RGB}{0, 43, 89}
\definecolor{tecnmWhite}{RGB}{255, 255, 255}

% Definir comandos para reducir la carga
\newcommand{\smallgray}[1]{{\scriptsize\textcolor{gray}{#1}}}

\fancyhead[L]{%
	\hspace*{0.5cm}\raisebox{-2.2\height}{\includegraphics[height=1cm]{imagenes/logotecnm.png}}%
}

\fancyhead[R]{%
	\raisebox{-1.7\height}{%
		\begin{tabular}[b]{r}
			{\scriptsize\textbf{Instituto Tecnológico de Morelia}}\\[-0.6em]
			\smallgray{Subdirección Académica}\\[-0.6em]
			\smallgray{Departamento de Sistemas y Computación}\\[-0.6em]
			\smallgray{Infraestructura de Servicios}\\[-0.4em]
			{\scriptsize Página \thepage\ de \pageref{LastPage}}
		\end{tabular}%
	}%
}

\makeatletter
\renewcommand\@biblabel[1]{#1.}
\makeatother

\fancyfoot[L]{%
	\small
	Tarea 4: Consolidación de Servidores
	
	\vspace{0.1cm}
	
	\small\textcolor{gray}{Mario Eduardo Sánchez Mejía}
}

\fancyfoot[R]{%
	\small
	\textbf{21120721}%
}

% Definir comando para citas
\newcommand{\cita}[1]{\cite{#1}}

\begin{document}
	\pagenumbering{arabic}
	
	
	% Secciones principales
	% portada.tex
% Configuración del fondo solo para la portada
\backgroundsetup{
	scale=1,
	angle=0,
	opacity=0.2,
	contents={\includegraphics[width=\paperwidth,height=\paperheight]{imagenes/fondo_portada.png}}
}
\BgThispage

\begin{center}
	\vspace*{2cm}
	
	\Large\textbf{Tarea 2: Primera Extracción de Datos}
	
	\vspace{1.5cm}
	
	Autor(es):
	
	\vspace{0.5cm}
	
	Luis Fernando Chávez Martínez
	
	Mario Eduardo Sánchez Mejía
	
	
	\vspace{1.5cm}
	
	Docente:
	
	\vspace{0.5cm}
	
	
	Aurelio Amaury Coria Ramirez
	
	\vspace{0.5cm}
	
	\begin{abstract}
		\noindent
		\justifying
		La presente investigación aborda el desarrollo de una alternativa de código abierto a las plataformas comerciales de Business Intelligence, centrándose específicamente en replicar las funcionalidades principales de Power BI utilizando Python. Se analiza la arquitectura necesaria para implementar un sistema completo de análisis y visualización de datos, identificando los componentes críticos para el procesamiento, análisis y presentación de información. La investigación examina las diferentes tecnologías y frameworks disponibles para cada componente del sistema, desde la extracción y transformación de datos hasta la visualización interactiva y el despliegue en producción. Este estudio contribuye al campo de la analítica de datos proporcionando un marco de trabajo para la implementación de soluciones personalizadas de Business Intelligence, beneficiando a organizaciones que requieren mayor flexibilidad y control sobre sus herramientas de análisis de datos. \\
		
		\textbf{Palabras clave:} ETL, Visualización de datos, Python, Desarrollo web, Bases de datos, Machine Learning, DevOps, Dashboards interactivos, Análisis de datos, Interfaz de usuario, Business Intelligence, Código abierto.
	\end{abstract}
\end{center}

% Desactivar el fondo para las siguientes páginas
\clearpage
\backgroundsetup{contents={}}
	
	
	\tableofcontents
	\newpage
	\section{Introducción}
\noindent
\justifying
Los sensores de movimiento basados en tecnología infrarroja pasiva (PIR) representan una herramienta fundamental en los sistemas de automatización y seguridad modernos. Estos dispositivos aprovechan la capacidad de detectar la radiación infrarroja emitida naturalmente por los cuerpos humanos y animales para identificar su presencia en un área determinada.

El presente trabajo se centra en la implementación de un detector de movimiento utilizando el sensor HC-SR501, un componente ampliamente utilizado en proyectos de electrónica debido a su bajo costo, facilidad de uso y confiabilidad. Este sensor opera detectando cambios en los patrones de radiación infrarroja dentro de su rango de visión, lo que permite activar diversos dispositivos o sistemas cuando se detecta movimiento.

La integración de este sensor con una Raspberry Pi permite crear un sistema inteligente capaz de reaccionar a la presencia humana. En este proyecto específico, el sistema activará un LED cuando detecte movimiento y lo desactivará cuando no haya presencia en el área de detección. Esta funcionalidad básica ilustra el principio operativo que sustenta aplicaciones más complejas como sistemas de iluminación automática, alarmas de seguridad, o contadores de personas.

El objetivo principal de esta práctica es comprender los principios de funcionamiento del sensor PIR, aprender a configurar sus parámetros de sensibilidad y tiempo de activación, e implementar un circuito funcional que demuestre su aplicación práctica. Adicionalmente, se busca familiarizar al estudiante con la programación de GPIO en Raspberry Pi para la lectura de sensores y control de actuadores.
	\newpage
	\section{Desarrollo}

\subsection{Extracción, Transformación y Carga de Datos (ETL)}

\subsubsection{Pandas}
Pandas es una de las bibliotecas más utilizadas en Python para la manipulación y análisis de datos. Proporciona estructuras de datos flexibles como DataFrame y Series, facilitando la limpieza, transformación y preparación de datos \cite{mckinney2010pandas}.

\begin{itemize}
	\item Manipulación de datos tabulares.
	\item Soporte para archivos CSV, Excel, JSON y bases de datos SQL.
	\item Métodos eficientes para filtrado, agregación y procesamiento de datos.
\end{itemize}

\subsubsection{NumPy}
NumPy proporciona soporte para arrays multidimensionales y funciones matemáticas avanzadas. Es fundamental para cálculos numéricos eficientes \cite{harris2020array}.

\begin{itemize}
	\item Soporte para operaciones matriciales.
	\item Funciones matemáticas optimizadas.
	\item Base para otras bibliotecas como Pandas y Scipy.
\end{itemize}

\subsubsection{PySpark}
PySpark es una interfaz de Apache Spark para Python, diseñada para procesar grandes volúmenes de datos en entornos distribuidos \cite{zaharia2016apachespark}.

\begin{itemize}
	\item Procesamiento de datos en paralelo.
	\item Manejo de datos en clústeres.
	\item Integración con otras herramientas de Big Data.
\end{itemize}

\subsubsection{SQLAlchemy}
SQLAlchemy es un ORM (Object-Relational Mapping) que permite interactuar con bases de datos SQL de manera eficiente \cite{bayer2010sqlalchemy}.

\begin{itemize}
	\item Abstracción de consultas SQL en Python.
	\item Conexión a múltiples bases de datos.
	\item Optimización de consultas y transacciones.
\end{itemize}

\subsubsection{Requests}
Requests es una biblioteca en Python utilizada para hacer solicitudes HTTP de manera sencilla y eficiente \cite{reitz2011requests}.

\begin{itemize}
	\item Extracción de datos desde APIs REST.
	\item Manejo de autenticación y sesiones.
	\item Soporte para solicitudes GET, POST, PUT y DELETE.
\end{itemize}

\subsubsection{BeautifulSoup y Scrapy}
BeautifulSoup y Scrapy son herramientas populares para la extracción de datos desde páginas web \cite{richardson2007beautifulsoup}.

\begin{itemize}
	\item Scrapy permite la automatización del web scraping.
	\item BeautifulSoup facilita la manipulación del HTML y XML.
	\item Soporte para navegación y extracción estructurada de datos.
\end{itemize}

\subsection{Visualización de Datos}

\subsubsection{Matplotlib}
Matplotlib es una biblioteca para la creación de visualizaciones estáticas en Python. Proporciona una interfaz similar a MATLAB para generar gráficos \cite{hunter2007matplotlib}.

\begin{itemize}
	\item Creación de gráficos de líneas, barras y dispersión.
	\item Personalización de ejes, colores y etiquetas.
	\item Exportación en formatos PNG, SVG y PDF.
\end{itemize}

\subsubsection{Seaborn}
Seaborn está basado en Matplotlib y permite la creación de gráficos estadísticos con mejor diseño y mayor facilidad \cite{waskom2021seaborn}.

\begin{itemize}
	\item Visualizaciones avanzadas de datos categóricos y numéricos.
	\item Integración con Pandas para análisis rápido.
	\item Gráficos de correlación y distribución de datos.
\end{itemize}

\subsubsection{Plotly}
Plotly es una biblioteca para la generación de visualizaciones interactivas, adecuada para dashboards y aplicaciones web \cite{plotly2021}.

\begin{itemize}
	\item Creación de gráficos 3D y animaciones.
	\item Integración con Dash para dashboards interactivos.
	\item Soporte para mapas geoespaciales y diagramas avanzados.
\end{itemize}

\subsubsection{Bokeh}
Bokeh permite la generación de gráficos interactivos con alto rendimiento \cite{bokeh2018}.

\begin{itemize}
	\item Creación de dashboards en navegadores web.
	\item Visualización de grandes volúmenes de datos.
	\item Personalización avanzada de gráficos.
\end{itemize}

\subsubsection{Dash}
Dash es un framework basado en Flask que permite la creación de dashboards interactivos \cite{plotly2017dash}.

\begin{itemize}
	\item Construcción de aplicaciones web con visualizaciones.
	\item Integración con Plotly para interactividad.
	\item Ideal para compartir análisis de datos.
\end{itemize}


	\newpage
	\section*{Conclusión}

La implementación del detector de movimiento con sensor infrarrojo PIR HC-SR501 y Raspberry Pi ha demostrado ser un proyecto exitoso que ilustra los principios fundamentales de los sistemas de detección automática. A través de este trabajo, se ha logrado comprender el funcionamiento interno del sensor PIR, sus modos de operación y las consideraciones necesarias para su implementación efectiva.

El sistema desarrollado presenta diversas ventajas como su bajo consumo energético, fácil configuración y alta sensibilidad para detectar presencia humana. Sin embargo, también se identificaron algunas limitaciones inherentes al sensor, como su dificultad para detectar movimiento perpendicular a la superficie del domo y la necesidad de un período de inicialización de aproximadamente un minuto.

Las aplicaciones potenciales de este tipo de sistema son numerosas y abarcan desde la automatización doméstica hasta sistemas de seguridad. La posibilidad de ajustar parámetros como la sensibilidad y el tiempo de retardo mediante los potenciómetros R1 y R2 brinda flexibilidad para adaptar el dispositivo a diferentes entornos y requisitos.

Para trabajos futuros, sería interesante expandir este proyecto incorporando múltiples sensores para cubrir áreas más amplias, integrar comunicación inalámbrica para notificaciones remotas, o combinar el sensor PIR con otras tecnologías como ultrasonido o visión por computadora para aumentar la precisión y reducir los falsos positivos.

En conclusión, este proyecto no solo ha permitido desarrollar habilidades prácticas en electrónica y programación, sino que también ha proporcionado una base sólida para comprender los principios fundamentales de los sistemas de detección automática, elementos esenciales en la creciente industria de la Internet de las Cosas (IoT) y la automatización inteligente.
	
	% Referencias
	%\clearpage
	%\addcontentsline{toc}{section}{Referencias}
	%\bibliography{referencias} 
	
\end{document}